Spontaneous formation of small-scale microstructures can have a deleterious effect on electron bunch stability and emission properties, and they are at the same time a tremendous source of coherent radiation in the terahertz domain3,4,5,6,7,8,9,10,11,12,13,14,15,16,17, provided the instability can be mastered. This is the reason why understanding and controlling the interplay between Coherent Synchrotron Radiation (CSR) and the microbunching instability has nowadays become a central open question in the development of synchrotron radiation facilities.

To answer this question, it is essential to develop ultrafast photonic devices for electron bunch shape characterization. The challenges for the photonics community is high, given the need for ultrashort (picosecond or femtosecond) temporal resolution, single-shot operation, at high repetition rates (MHz and more), and given the particularly challenging environment near relativistic electron bunches. Recent advances consequently pushed photonics systems beyond the state of the art. Ultrafast electric-field measurement techniques using femtosecond laser pulses (electro-optic sampling24) have allowed single-shot bunch shape measurements (plural)25, and these techniques have then been extensively investigated and improved this last decade


The ULTRASYNC (Exploration et contrôle ULTRArapide de la dynamique des paquets d'électrons dans les sources de lumière SYNChrotron, ANG-DFG) aims to study and control coherent synchrotron radiation (CSR) and microbunching instability. The goal is to obtain a high power and stable coherent emission (KARA and SOLEIL) by feedback control. At SOLEIL, this was already succesfully achieved, still there is a significant limit to the range, where this concept works. One possible way would be to calculate the necessary feedback from a whole THz CSR pulse shape, which would require to record an electro-optical signal (time-stretch). This leads to a possible approach by combining time-stretch and FPGA. 

For the moment, the only experimental test has been made using a relatively simple feedback:
- a bolometer signal as the feedback loop input
- a low-cost FPGA (redpitaya) that sends the feedback on the accelerating voltage
There are limitations in the maximal bunch charge in the accelerator. So an open question is whether measuring each THz pulse using EO sampling + time-stretch may help to solve this open problem. In clear:
- EO sampling + time-stretch as in Eléonore's thesis
-> association with the new FPGA-based system
-> finding adequate feedback that is programmed in the FPGA
may solve the problem, and allow the control to succeed at the highest currents in SOLEIL. 
Target would be ~15 mA for 1 bunch (and feedback control presently works to a little more than ~10 mA).

At 1 micron wavelength, the maximum stretch is limited due to fiber glass properties (at this wavelength). The maximum length achieved are 2-4 ns (SOLEIL and KARA). For this reason, a fast ADC is critical, to get a good ENOB/number of sampling points. A relevant figure of merit (that characterizes the effective number of points) is -- we think: FOM=electronic bandwidth x stretch time. There is some bottleneck concerning time-stretch at this wavelength, and one way to solve the issue is to try pushing the bandwidth of ADCs.

At 1550 nm wavelength, it is possible to stretch the signal up to tens of nanoseconds. Therefore, a relatively cheap ADC would already be sufficient and allow an effective number of samples at around 100. Applications still need to be investigated.

As potential applications for the new system would be applications for accelerators in the 1 micron wavelength (best concerning EO sampling SNR and bandwidth).

