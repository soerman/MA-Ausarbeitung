In many scientific applications and experiments, the observation of non-repetitive, statistically rare events with very high occurrence rates is desired.
As these events might occur on a time scale of femtoseconds, real-time measurement systems with fine temporal resolution are needed that are also capable of long acquisition times.
This imposes high technological challenges on \glspl{daq} and \glspl{adc}.

One bottleneck in the acquisition of ultra-fast events is the limited timing performance of commercially available \glspl{adc}. 
The limitation posed by the converters is a trade-off between the dynamic range (\gls{enob}) and sampling rate of the converters.
With an increase in the the sampling rate, ambiguity of the comparators (output neither '0' nor '1') inside the \gls{adc} and sampling errors due to clock jitter become major limiting factors on the overall performance. \cite{Mahjoubfar2017}

A first demonstration of a concept to overcome these limitations was presented in 1999 by \cite{ts_adc}. 
The idea is to stretch the analog signal in time before digitizing it in the converter and hence relax the demands on the data converter performance. 
This time-stretching is accomplished by using chirped optical pulses and chromatic dispersion in optical fibers.
The concept is therefore called ``photonic time-stretch'' and was successfully tested in combination with a moderate-speed \gls{adc} in \cite{ts_adc}.

Since then, the time-stretch method has been continuously improved and has found use in many applications.
For example, in biomedical diagnostics, a first demonstration of an artificial intelligence based high-speed phase microscope has been developed. 
It uses \gls{tsqpi}, a technique based on the time-stretch concept which enables simultaneous measurement of phase and spatial intensity profiles.
This allows label-free classification of cells for cancer diagnostics and drug development. \cite{Mahjoubfar2017} 

The time-stretch concept is also useful for applications in particle accelerators due to the short timescales involved.
In a storage ring for example, relativistic electron bunches interact with their own radiation which can lead to the formation of spatial microstructures inside the bunches, a phenomenon also called micro-bunching instability.
This is a source of intense pulses of terahertz radiation, the so called \gls{csr}. 
Due to its non-ionizing nature, this radiation can be potentially used in imaging applications for many research fields such as biology and is therefore an important field of study. 
A first demonstration of direct observation of the micro-bunching instability was performed at the synchrotron facility \gls{soleil} using a time-stretched signal together with a real-time oscilloscope. \cite{roussel2015}

The use of the time-stretch method in different applications has demonstrated the advantages to measure events with femtosecond resolution.
However, commercially available real-time diagnostics systems are limited in memory space. 
The current maximal memory depth of such systems lies in the the range of few \si{\giga \sample}.
This limits the acquisition time of such systems at maximum sampling rate, which lies in the range of a few milliseconds at best.
Due to this limitation it is not possible to measure data continuously over a long period of time. 
This creates a problem in applications where a longer observation time (up to hours) is required, e.g. in accelerator applications where the turn-by-turn (and bunch-by-bunch) analysis of the electron bunches is desired in order to study the evolution of the bunch profiles. 

This challenge was the motivation to design novel ultra-fast acquisition systems based on the photonic time-stretch \gls{adc}. 
Together with the next generation of \gls{fpga}-based systems with integrated high-performance \glspl{adc}, this gives rise to a new concept of \gls{daq}, the photonic time-stretch \gls{daq}.
The photonic time-stretch \gls{daq} consists of a photonic part, containing the time-stretching section and the conversion of photons into electrical signal with a photo-detector. 
Furthermore, such a system has one or multiple \glspl{adc} converting the analog signals into digital samples.
The digital samples are then processed in a computing unit and broadcast to other units as needed if the system is integrated e.g. into a cluster of distributed instrumentation systems. 
 

\section{Objective}
In this thesis, a first demonstrator of a \gls{daq}-system based on the time-stretch concept is developed.
This system, called \gls{theresa}, enables high-speed measurements of ultrafast events with a time resolution in the range of femtoseconds.

In order to achieve such high resolution, the time-stretch technique will be used in order to stretch the input signal in the range of pico- to nano-seconds.
The input signal is continuously sampled by high-speed \glspl{adc} with a temporal resolution defined by the user as needed.
To sample the signal, the \glspl{adc} need to have a sampling rate in the order of several \si{\GHz}.
The amplitudes of the signals to be measured are very small and an \glspl{adc} with an appropriate resolution has to be chosen in order to guarantee an \gls{enob} of at least \SI{10}{\bits}. \cite{bielwaski}

This leads to the next challenge: Sampling at several \si{\GHz} with high resolution implies a large amount of data, leading to a data rate in the range of terabits per second.
In order to enable such a high data-throughput, the system will be based on a new generation of \gls{soc}, integrating a \gls{fpga} and a processing unit together with the high-speed \glspl{adc}. 
The \gls{soc} will have high-speed peripherals in order to guarantee the continuous high-speed data-throughput. 
Combination with the \gls{fpga} should allow for flexible system tuning for a user-defined application.
The user will be able to control and configure the system via an application or operating system running on the processing unit.

Furthermore, the system should be compatible with already existing high-speed \gls{daq} frameworks (e.g. based on \gls{pcie}) and be easily integrated into the system for the user application (e.g. through optical fibers to a distributed instrumentation system). 
However, stand-alone operation should also be possible.
Furthermore, the \gls{daq} should be designed in such way that usage independently from the time-stretch method is possible.

The overall thesis is structured in the following way: 
In \autoref{chap:motivation} necessary theoretical background for the new \gls{theresa} system is given. 
The subject of \gls{thz} science in particular is touched being the main motivation for the design of the novel time-stretch sampling system.
The general architecture of \gls{theresa} is covered in \autoref{chap:new_sys}, including also state of the art readout-systems, especially \gls{kapture} which is in operation at \gls{kara}.
The design steps of the front-end sampling card of \gls{theresa} are described in \autoref{chap:samplingboard} in detail. 
The description of the back-end readout card, as well as an overview of the design of appropriate firmware is covered in \autoref{chap:readout}.
At last, results are concluded and an outlook for the newly developed system is given.

\glsresetall