\section{Coherent Synchrotron Radiation}
Synchrotron radiation (SR) is produced in synchrotron radiation facilites (like electron storage rings) by accelerating relativistic electrons. Emission of  SR occurs, when bending/deflecting the electrons in dipole magnets or undulators\footnote{Undulators are used to make the electrons oscillate by generating a periodic magnetic field}. 

\autoref{fig:storageRing} shows a general scheme of an electron storage ring. Electrons, or rather electron bunches, are generated with an electron gun and are accelerated to almost speed of light by a linear accelerator (LINAC). The bunches are injected into the storage ring, after reaching their nominal energy in a booster. In the ring, the path of the electron bunches is altered by bending magnets, leading them on a circular trajectory. Due to emission of SR at each bending, the electrons lose energy, which has to be compensated. This is done in an accelerating radiofrequency (RF) cavity. Not shown in the scheme are the beamlines, which lead the SR radiation, or rather chosen wavelength regions of it, through an optical system to the respective user experiments.\cite{roussel2014} \cite{rota2018}

 \begin{figure}[H]
 	\centering
 	\includegraphics[width=0.7\textwidth]{chap/02-theory/img/synchrotron}
 	\caption{Basic scheme of an electron storage ring (redrawn from \cite{roussel2014})}
 	\label{fig:storageRing}
 \end{figure}
The range of SR reaches from hard X-rays down to the infrared region of the electromagnetic spectrum (see \autoref{fig:spectrum}). In contrast to other sources, it has properties like:
\begin{itemize}[noitemsep]
	\item high intensity 
	\item high collimation
	\item polarisation
	\item well-defined timing of pulses
\end{itemize}

Due to this properties, synchrotrons are used for microscopy, spectroscopy, and time-resolved experiments in such fields like condensed matter physics, biology, material science and many more. 
\begin{figure}[H]
	\centering
	\includegraphics[width = 0.8\textwidth, height = 0.5\textwidth]{chap/02-theory/img/spectrum.tikz}
	\caption{Electromagnetic spectrum}
	\label{fig:spectrum}
\end{figure}

\subsection{Micro-Bunching Instabilities}
As demands are ever-increasing, electron accelerators need to provide higher brilliance (or brightness). This is achieved by increased photon flux and reduction of the transverse emittance. For longidutinal coherence, the electron bunches are shortened, which results in emission of coherent synchrotron radiation (CSR) at frequencies up to the THz range. However, this introduces complex dynamics, as the electrons interact with their own radiation. This manifests into the so called microbunching instability: the formation of microstructures (in the sub-millimeter to centimeter range) in the longitudinal density profile of the electron bunches. Being on the one side a limitation to the stable operation of the overall system at high current density/short bunch length mode. On the other side, these instabilities can be potential sources of brilliant THz radiation. A thorough understanding of these dynamics is necessary to control the emission in this spectral domain which enables usage in experiments. \cite{rota2018} \cite{brosi}

\begin{figure}[H]
	\centering
	\includegraphics[width = 0.7\textwidth]{chap/02-theory/img/csr2.png}
	\caption{Placeholder \cite{rota2018}}
	\label{fig:csr}
\end{figure}

\begin{figure}[H]
	\centering
	\includegraphics[width = 0.5\textwidth]{chap/02-theory/img/microbunching}
	\caption{Electrons interact with their own radiation \cite{Bielawski2019}}
	\label{fig:microBunch}
\end{figure}









\paragraph{KARA}
\begin{itemize}[noitemsep]
	\item Located at the Karlsruhe Institute of Technology (KIT)
	\item Up to 184 electron packages (bunches) can be filled with a distance of \SI{2}{\nano \second} (\SI{500}{\mega \hertz}) between two adjacent bunches 
	\item Operated by the Institute of Beam Physics and Technology (IBPT)
	\item Microtron, Booster Synchrotron, and Storage Ring
\end{itemize}

\begin{table}[tbh!]
	\caption{KARA characteristics}
	\label{tab:kara}
	\begin{minipage}{\textwidth}
		\centering
		\begin{tabularx}{\textwidth}{XS}
			\toprule
			\textbf{Beam energy}     				& $ \SI{2.5}{\giga \electronvolt}$ \\
			\textbf{Circumference} 	 				& $\SI{110}{\meter}$	  \\
			\textbf{RF frequency }   				& $\SI{499.7}{\mega \hertz}$ 	\\
			\textbf{Harmonic number} 				& 184	\\
			\textbf{Number of RF stations} 			& 2 \\
			\textbf{Number of cavities per station} 	& 2	\\
			\textbf{Accelerating voltage} 					& $\SI{1.4}{\mega \electronvolt}$ \\
			\bottomrule		\end{tabularx}
	\end{minipage}
\end{table}



\begin{figure}[H]
	\centering
	\includegraphics[width = 0.6\textwidth]{chap/02-theory/img/kara.png}
	\caption{Facility \cite{rota2018}}
	\label{fig:kara}
\end{figure}

\newpage 
\section{Photonic time-stretch analog-to-digital converter}










"' In recent and future synchrotron radiation facilities, relativistic electron bunches with increasingly high charge density are needed for producing brilliant light at various wavelengths, from X-rays to terahertz. In such conditions, interaction of electron bunches with their own emitted electromagnetic fields leads to instabilities and spontaneous formation of complex spatial structures. Understanding these instabilities is therefore key in most electron accelerators. However, investigations suffer from the lack of non-destructive recording tools for electron bunch shapes. In storage rings, most studies thus focus on the resulting emitted radiation. Here, we present measurements of the electric field in the immediate vicinity of the electron bunch in a storage ring, over many turns. For recording the ultrafast electric field, we designed a photonic time-stretch analog-to-digital converter with terasamples/second acquisition rate. We could thus observe the predicted link between spontaneous pattern formation and giant bursts of coherent synchrotron radiation in a storage ring. "'  \cite{Bielawski2019}

\begin{figure}[H]
	\centering
	\includegraphics[width = 0.8\textwidth]{chap/02-theory/img/EO.png}
	\caption{Electro-Optical Time-Stretch Technique \cite{Bielawski2019}}
	\label{fig:eo}
\end{figure}


\subsection{Applications for Photonic Time-Stretch}

\newpage


\section{Analog-To-Digital Converters}
The following section aims to recapitulate the theory and characteristics concerning Analog-To-Digital-Converters (ADCs).
\subsection{Sampling Theory}
An ADC samples an analog signal with a frequency $f_s$. This frequency has to be chosen in such way, that the original signal can be fully reconstructed. The \textit{Nyquist criteria} states, that in order to accurately represent a bandlimited, continous signal
\begin{equation}
	y (t) \, \fourier  Y(f) \quad \text{with} \quad Y(f) = 0, \, |f| \geq \frac{B}{2}
\end{equation}
it has to be sampled with a frequency $f_s$ respecting
\begin{equation}
	f_s \geq B \quad \text{or} \quad f_s \geq 2 f_a
\end{equation}
with $f_a$ being the highest frequency contained in the signal. \cite{walt} \cite{puente2015} \\
In other words, $f_a$ must be inside of the \textit{Nyquist bandwidth}, which is the spectrum from DC to $f_s/2$. Violation of this rule leads to \textit{aliasing}.

\begin{figure}[H]
	\centering
	\begin{subfigure}{\textwidth}
	\centering
	% include first image
	\includegraphics[width=.6\linewidth]{chap/02-theory/img/alias_f}  
	\caption{Sampling in frequency domain}
	\label{fig:alias_f}
	\end{subfigure}
	\begin{subfigure}{\textwidth}
	\centering
	% include first image
:alias	\includegraphics[width=.6\linewidth]{chap/02-theory/img/alias_t}  
	\caption{Aliasing in time domain}
	\label{fig:alias_t}
	\end{subfigure}
	\caption{Analog signal with frequency $f_a$ sampled at $f_s$ respecting (A) and not respecting (B) the Nyquist criteria. \autoref{fig:alias_t} shows the effect of case B in time domain. \cite{walt}}
\end{figure}


\paragraph{Sample-And-Hold-Amplifier}
ADCs need a certain amount of time to sample the input signal. If the value of the signal changes by more than one Least-Significant-Bit (LSB) during this period, this can result in large errors in the output signal. Therefore, so called Sample-And-Hold-Amplifiers (SHA) are used in front of ADCs to hold the  input value for the needed amount of time. The ADC sampling time needs to be timed in such way, that the analog-to-digital conversion falls into the hold period of the SHA and doesn't exceed into the sample period, for example like shown schmeatically in the diagram below. Thus, the upper frequency limitation is not determined by the ADC itself, but rather by the aperture jitter,bandwidth, distortion, etc. of the SHA. \cite{walt}



\begin{figure} [H]
	\centering
	\tikzexternaldisable
	\begin{tikztimingtable}
		[%
		timing/dslope=0.1,
		timing/name/.style={font=\sffamily\normalsize},
		timing/d/text/.style={font=\sffamily\normalsize},
		grayz/.style={timing/z/.append style={gray}},
		timing/n/.style={rectangle},
		timing/metachar={{K}[2]{#1l !{++(0,+.5\yunit)} N[rectangle,scale=.6]{\shortstack{#2}} !{++(0,-.5\yunit)} #1l}},
		timing/metachar={{J}[2]{#1h !{++(0,-.5\yunit)} N[rectangle,scale=.6]{\shortstack{#2}} !{++(0,+.5\yunit)} #1h}},
		]
		SHA & 1H 8K{HOLD} 8J{SAMPLE} 8K{HOLD} 3H\\
		Sampling & 5S A 15S A                    \\
	\end{tikztimingtable}
	\tikzexternalenable
\end{figure}

In addition to the SHA, there is also the Track-And-Hold-Amplifier (THA). Instead of a sample period, the THA has a track period, where the output of the amplifier tracks the input signal (see also \autoref{fig:tha}). When switching to hold mode, the signal at this instant is held. This is opposed to the SHA, where the output during sample mode is actually not defined and is set to the value of the input signal, only when switching into hold mode. 

\begin{figure}[H]
	\centering
	\includegraphics[width = 0.5\textwidth]{chap/02-theory/img/tha}
	\caption{Track-And-Hold-Amplifier schematic and principle \cite{walt}}
	\label{fig:tha}
\end{figure}

\subsection{Characteristics of Analog-To-Digital-Converters}
ADCs are used to translate analog quantities into digital signals, which can be processed by information processing, computing, data transmission and control systems. The translation can be seen as encoding a continuous-time analog input (voltage) into a series of discrete, N-bit words. This can be experessed as
\begin{equation}
	V_{\text{IN}} = V_{\text{FS}} \sum_{k = 0}^{N-1} \frac{b_k}{2^{k+1}} + \epsilon
\end{equation}
with $V_{\text{IN}}$ being the input voltage, $V_{\text{FS}}$ the full-scale voltage, $b_k$ the individual output bits and $\epsilon$ the quantization error. \autoref{fig:idealADC} shows the ideal transfer function of a 3-bit ADC.
\begin{figure}[H]
	\centering
	\includegraphics[width = 0.55\textwidth]{chap/02-theory/img/ideal_adc}
	\caption{Transfer function of an ideal, 3-bit ADC (redrawn from \cite{Lundberg})}
	\label{fig:idealADC}
\end{figure}



For an ideal converter, the number of bits would be sufficient to fully characterize it. Real ADCs however differ from the ideal behaviour, introducing static and dynamic imperfections. Different applications have different requirements, which leads to a number of specifications. These can be divided into three categories \cite{Lundberg}:
\begin{itemize}[noitemsep]
	\item Static parameters
	\item Frequency-domain dynamic parameters
	\item Time-domain dynamic parameters
\end{itemize}
This section provides an overview of these figures of merit. Which of these are needed to specify the necessary performance of the ADC has to be chosen for each application accordingly.

\subsubsection*{Static parameters}
\textit{Static parameters} are specifications, which can be measured at low speed/DC. 
\paragraph{Accuracy}
\textit{Accuracy} is the total error at a known voltage, which includes:
\begin{itemize}[noitemsep]
	\item Quantization error
	\item Gain error
	\item Offset error
	\item Nonlinearities
\end{itemize}

\paragraph{Resolution}
\textit{Resolution} is the number of bits $N$ of the ADC. Depending from the resolution are the size of the LSB, which in its turn determines the dynamic range, code widths and quantization error.

\paragraph{Dynamic Range}
Ratio between smallest possible output (LSB voltage) and the largest possible output (full-scale voltage). It cn be calculated as
\begin{equation}
	20 \log 2^{N} \approx 6N
\end{equation}


\paragraph{Offset and Gain Error}
The \textit{offset error} is the deviation of the first transition voltage from the ideal $1/2$ LSB. \textit{Gain Error} defines the deviation of the slope of the line going through the zero and full-scale point of the transfer function. These errors can easily be corrected by calibration. Refer to \autoref{fig:offsetErr}

\begin{figure}[H]
	\centering
	\includegraphics[width = 0.55\textwidth]{chap/02-theory/img/offset_err.tikz}
	\caption{Offset and Gain Error in the ADC characteristic. Notice the difference between the ideal, dotted line}
	\label{fig:offsetErr}
\end{figure}

\subsubsection*{Dynamic performance}
In an ADC (with built-in SHA) there are a couple of sources, which introduce noise and distortion:
\begin{itemize}
	\item \textbf{Input Stage:} Wideband noise, nonlinearity and bandwidth limitation
	\item \textbf{SHA:} Nonlinearity, aperture jitter\footnote{\textit{Aperture jitter} is the sample-to-sample variation in aperture delay, with \textit{aperture delay} meaning the time, which is needed by the SHA to disconnect the holding capacitor from the input buffer.} and bandwidth limitation
	\item \textbf{ADC:} Quantization noise, integral and differential nonlinearity
\end{itemize}
In the following the most important specifications are described, which are used to characterize the performance of ADCs.

\paragraph{Quantization Noise}
Even in an ideal N-bit converter there will be errors during the quantization, which behave like noise. The reason is that each N-bit word represents a certain range of analog input values, which is 1 LSB wide (\textit{code width}) and centered around a \textit{code center} (see \autoref{fig:idealADC}) \cite{Lundberg}. The input voltage is always assigned to the word of the nearest code center. This means that there will always be a difference between the code center and the actual input. The difference between the corresponding voltage of the respective code center and the analog input is called the \textit{quantization error}. For an equidistant quantization it is
\begin{equation}
	\left| e_q(t) \right| = \left| x(t) - x_q(t) \right| \leq \frac{q}{2}
\end{equation}
with $x_q(t)$ being the quantized/discrete signal, $x(t)$ the input signal and $q$ the width of the quantization stage. \cite{puente2015} 

Assuming the error voltage uncorrelated and uniformly distributed, the theoretical (maximum) Signal-To-Noise-Ratio (SNR) of this \textit{quantization noise} can be calculated. In the time domain, the quantization error can be approximated with a sawtooth signal:
\begin{equation}
	e(t) = st, \quad -\frac{q}{2s} < t < \frac{q}{2s} 
\end{equation}
\begin{figure}[H]
	\centering
	\includegraphics[width = 0.5\textwidth]{chap/02-theory/img/quantization_error.tikz}
	\caption{Quantization noise as function of time (redrawn from \cite{walt})}
	\label{fig:eq}
\end{figure}

The power of the quantization noise, which is assumed to be uncorrelated and broadband, can be calculated as the mean-square of $e(t)$:
\begin{equation}
	P_{QN} = e_{\text{rms}}^{2} = \overline{e^{2}(t)} = \frac{s}{q}\int_{-q/2s}^{+q/2s} (st)^{2} dt = \frac{s^3}{q} \left[ \frac{t^3}{3}\right]_{-\frac{q}{2s}}^{+\frac{q}{2s}} = \frac{q^2}{12}
\end{equation}

To calculate the maximal SNR of an ideal converter, a full-scale input sine wave is assumed:
\begin{equation}
	u(t) = u_s \sin(2\pi f t) = \frac{2^{N}q}{2}\sin(2\pi f t)  = 2^{N-1}q \sin(2\pi f t)
\end{equation}
With the effective value of the signal amplitude
\begin{equation}
	u_{\text{eff}} = \frac{u_s}{\sqrt{2}} = \frac{2^{N-1}q}{\sqrt{2}}
\end{equation}
and the quantization noise power, the SNR can be calculated as
\begin{equation}
	\text{SNR} = \frac{P_{\text{signal}}}{P_{\text{noise}}} = \frac{u_{\text{eff}}^{2}}{e_{\text{rms}}^{2}} = \frac{2^{2N-2}q^2/2}{q^2/12} = 2^{2N} \cdot 1.5.
\end{equation}
In decibel:
\begin{equation}
	\text{SNR}|_{\text{dB}} = 10\log\left(2^{2N}\cdot 1.5\right) = 6.02 N + 1.76
\end{equation}
\cite{puente2015} \cite{walt}

\paragraph{Equivalent Input Referred Noise}
Internal circuits of wideband ADCs produce rms noise due to resistor and thermal ("kT/C") noise, which is also present for DC signals. Therefore, the output of the ADC is a distribution of codes which is centered around the value of a DC input. For measuring the value of the noise, the ADC input is grounded, or held at a specific DC value, and a large amount of samples is collected and plotted as a histrogram. The noise is approximately Gaussian, thus the stndard deviation can easily be calculated.

\paragraph{Noise-Free Code Resolution}
The input referred noise described above determines the \textit{noise-free code resolution}, which is the number of bits, beyodn which it is not possible to resolve individual codes.

\paragraph{Integral and Differential Nonlinearity Distortion} 
\textit{Integral nonlinearity} in the transfer function of data converters results from the integral nonlinearities of the front-end, SHA and also the ADC itself. These nonlinearities depend on the input signal amplitude. distance of the code centers in the A/D converter characteristic from the ideal line
\textit{Differential nonlinearities} stem exclusively from the encoding process in the ADC. They not only depend on the input signal amplitude, but also on the positioning along the transfer function. The deviation of the code transition widths from the ideal width
of 1 LSB

\paragraph{Harmonic Distortion, Worst Harmonic, Total Harmonic Distortion (THD), Total Harmonic Distortion + Noise (THD + N)}

\begin{itemize}
	\item \textbf{Harmonic distortion (dBc\footnote{decibels below carrier}):} Input signal near full scale
\end{itemize}

\begin{itemize}
\item 
\item Signal-to-Noise-and-Distortion Ratio (SINAD, or S/N + D), Signal-to-Noise ratio (SNR), Effective Number of Bits (ENOB)
\item Analog Bandwidth (Full-Power, Small-Signal)
\item Spurious Free Dynamic Range (SFDR)
\item Two-Tone Intermodulation Distortion, Multi-Tone Intermodulation Distortion
\item Noise Power Ratio (NPR)
\item Adjacent Leakage Ratio (ACLR)
\item Noise Figure
\item Setting Time, Overvoltage Recovery Time
\end{itemize}

\cite{walt}
\subsection{Interleaving}

\begin{itemize}
\item Net sample rate
\item Interleaving Spurs
\end{itemize}

\cite{mangrob}




\newpage
\section{RF/Microwave Design Basics}

\subsection{General Techniques/Strategies?}

\subsection{Coplanar Waveguides}
\paragraph{Surface Coplanar Waveguide with Ground}  
\begin{figure}[!htbp]
	\centering
	\begin{tikzpicture}
		\filldraw[color=black, fill=black] (0,0.7) rectangle ++(9,0.3) node[pos=.5](gnd){};
		\filldraw[color=black, fill=gray!20] (0,1) rectangle ++(9,2) node[pos=.5]{\(\varepsilon_r\)};
		\filldraw[color=black, fill=black] (0,3) rectangle ++(2,.2) node[pos=.5](GND1){};
		\filldraw[color=black, fill=black] (3,3) rectangle ++(1,.2) node[pos=.5](cond1){};
		\filldraw[color=black, fill=black] (5,3) rectangle ++(1,.2) node[pos=.5](cond2){};
		\filldraw[color=black, fill=black] (7,3) rectangle ++(2,.2) node[pos=.5](GND2){};
		\draw[>=triangle 45, <->] (-0.5,1) -- (-0.5,3) node[pos=.5,anchor=west](){\(h\)};
		\draw[>=triangle 45, <->] (2,4) -- ++(1,0) node[pos=.5,anchor=south](){\(d\)};
		\draw[>=triangle 45, <->] (3,3.8) -- ++(1,0) node[pos=.5,anchor=south](){\(w\)};
		\draw[>=triangle 45, <->] (4,3.6) -- ++(1,0) node[pos=.5,anchor=south](){\(s\)};
		%\draw[>=triangle 45,  <->] (5.7,3) -- ++(0,0.2) node[pos=.5,anchor=west](){\(t\)};
		
		\draw[dashed] (9,3.2) -- (10,3.2);
		\draw[dashed] (9,3) -- (10,3);
		\draw[dashed] (0,1) -- (-1,1);
		\draw[dashed] (0,3) -- (-1,3);
		
		\draw[>=triangle 45, <->] (-0.5,1) -- (-0.5,3) node[pos=.5,anchor=west](){\(h\)};
		\draw[>=triangle 45, ->] (10,4) -- (10,3.2) node[pos=.5,anchor=west](){\(t\)};
		\draw[>=triangle 45, ->] (10,2.2) -- (10,3) node[pos=.5,anchor=west](){};
		
	\end{tikzpicture}
	\caption{Edge-Coupled Coplanar Waveguide}
	\label{fig:microstrip_geometry}
\end{figure}

The corresponding equations are \cite[~p197-198]{wadell}: 
\begin{equation}
	Z_{0,o} = \frac{\eta_0}{\sqrt{\epsilon_{eff,o}}} \left( \frac{1.0}{2.0 \frac{K(k_o)}{K'(k_o)} + \frac{K(\beta_1)}{K'(\beta_1)}}\right)
\end{equation}    

\begin{equation}
	Z_{0,e} = \frac{\eta_0}{\sqrt{\epsilon_{eff,e}}} \left( \frac{1.0}{2.0 \frac{K(k_e)}{K'(k_e)} + \frac{K(\beta_1 k_1)}{K'(\beta_1 k_1)}}\right)
\end{equation} 


\begin{equation}
	\epsilon_{eff,o} = \frac{2.0 \epsilon_r \frac{K(k_o)}{K'(k_o)} + \frac{K(\beta_1)}{K'(\beta_1)}}{2.0 \frac{K(k_o)}{K'(k_o)} + \frac{K(\beta_1)}{K'(\beta_1)}}
\end{equation}

\begin{equation}
	\epsilon_{eff,e} = \frac{2.0 \epsilon_r \frac{K(k_e)}{K'(k_e)} + \frac{K(\beta_1 k_1)}{K'(\beta_1 k_1)}}{2.0 \frac{K(k_e)}{K'(k_e)} + \frac{K(\beta_1 k_1)}{K'(\beta_1 k_1)}}
\end{equation}

Where

\begin{equation}
	k_o = \Lambda \frac{-\sqrt{\Lambda^2 - t_c^2} + \sqrt{\Lambda^2 - t_B^2}}{t_B\sqrt{\Lambda^2 - t_c^2} + t_c \sqrt{\Lambda^2 - t_B^2}}
\end{equation}

\begin{equation}
	k_e = \Lambda' \frac{-\sqrt{\Lambda'^2 - t_c'^2} + \sqrt{\Lambda'^2 - t_B'^2}}{t_B'\sqrt{\Lambda'^2 - t_c'^2} + t_c' \sqrt{\Lambda'^2 - t_B'^2}}
\end{equation}


\begin{equation}
	\Lambda = \frac{\sinh^2 \left( \frac{\pi (s/2.0 + w + d)}{2.0 h} \right) }{2}
\end{equation}

\begin{equation}
	t_c = \sinh^2 \left( \frac{\pi (s/2.0 + w)}{2.0 h} \right) - \Lambda
\end{equation}

\begin{equation}
	t_B = \sinh^2 \left( \frac{\pi s}{4.0 h} \right) - \Lambda
\end{equation}


\begin{equation}
	\Lambda' = \frac{\cosh^2 \left( \frac{\pi (s/2.0 + w + d)}{2.0 h} \right) }{2}
\end{equation}

\begin{equation}
	t_c' = \sinh^2 \left( \frac{\pi (s/2.0 + w)}{2.0 h} \right) - \Lambda' + 1.0
\end{equation}

\begin{equation}
	t_B' = \sinh^2 \left( \frac{\pi s}{4.0 h} \right) - \Lambda + 1.0
\end{equation}

The parameters have to be chosen according to 
\begin{equation}
	s + 2.0 w + 2.0 d \leq h
\end{equation}
to guarantee coplanar propagation. \cite{wadell}

\paragraph{Surface Coplanar Waveguide with Ground}  

\begin{figure}[!htbp]
	\centering
	\begin{tikzpicture}
		\filldraw[color=black, fill=black] (0,0.7) rectangle ++(9,0.3) node[pos=.5](gnd){};
		\filldraw[color=black, fill=gray!20] (0,1) rectangle ++(9,2) node[pos=.5]{\(\varepsilon_r\)};
		\filldraw[color=black, fill=black] (0,3) rectangle ++(2,.2) node[pos=.5](GND1){};
		\filldraw[color=black, fill=black] (3.5,3) rectangle ++(2,.2) node[pos=.5](cond1){};
		\filldraw[color=black, fill=black] (7,3) rectangle ++(2,.2) node[pos=.5](GND2){};
		\draw[>=triangle 45, <->] (3.5,3.4) -- ++(2,0) node[pos=.5,anchor=south](){\(a\)};
		\draw[>=triangle 45, <->] (2,3.8) -- ++(5,0) node[pos=.5,anchor=south](){\(b\)};
		
		
		\draw[dashed] (9,3.2) -- (10,3.2);
		\draw[dashed] (9,3) -- (10,3);
		\draw[dashed] (0,1) -- (-1,1);
		\draw[dashed] (0,3) -- (-1,3);
		
		\draw[>=triangle 45, <->] (-0.5,1) -- (-0.5,3) node[pos=.5,anchor=west](){\(h\)};
		\draw[>=triangle 45, ->] (10,4) -- (10,3.2) node[pos=.5,anchor=west](){\(t\)};
		\draw[>=triangle 45, ->] (10,2.2) -- (10,3) node[pos=.5,anchor=west](){};
		%\draw[>=triangle 45,  <->] (5.7,3) -- ++(0,0.2) node[pos=.5,anchor=west](){\(t\)};
	\end{tikzpicture}
	\caption{Coplanar Waveguide with Ground}
	\label{fig:microstrip_geometry}
\end{figure}


The characteristic impedance of a coplanar waveguide is given as follows \cite{wadell}: 
\begin{equation}
	Z_0 = \frac{60.0 \pi}{\sqrt{\epsilon_{eff}}} \frac{1.0}{\frac{K(k)}{K(k')} + \frac{K(k_1)}{K(k_1')}}
\end{equation}
It comprises of the following components, with K(k) being an elliptical integral of the first kind (see also \cite[p.~430]{bronstein}):
\begin{equation}
	k = a/b
\end{equation}
\begin{equation}
	k' = \sqrt{1.0 - k^{2}}
\end{equation}
\begin{equation}
	k_1' = \sqrt{1.0 - k_1^{2}}
\end{equation}
\begin{equation}
	k_1 = \frac{tanh(\frac{\pi a}{4.0  h})}{tanh(\frac{\pi  b}{4.0 h})}
\end{equation}
\begin{equation}
	\epsilon_{eff} = \frac{1.0 + \epsilon_r \frac{K(k')}{K(k)} \frac{K(k_1)}{K(k_1')}}{1.0 + \frac{K(k')}{K(k)} \frac{K(k_1)}{K(k_1')}}
\end{equation}