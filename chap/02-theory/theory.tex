\section{Synchrotron}
\begin{figure}[H]
	\centering
	\includegraphics[width = 0.8\textwidth]{chap/02-theory/img/spectrum.tikz}
	\caption{Electro-Magnetic spectrum}
	\label{fig:spectrum}
\end{figure}

\newpage
\subsection{KARA}

\begin{itemize}
	\item Located at the Karlsruhe Institute of Technology (KIT)
	\item Up to 184 electron packages (bunches) can be filled with a distance between two adjacent bunches of \SI{2}{\nano \second}
	\item Operated by the Institute of Beam Physics and Technology (IBPT)
\end{itemize}

\subsubsection{KAPTURE-2}
KAPTURE (\textbf{Ka}rlsruhe \textbf{P}ulse \textbf{T}aking \textbf{U}ltra-fast \textbf{R}eadout \textbf{E}lectronics) is a system -- integrated in KARA -- designed to continuously sample ultra-short pulses generated by terahertz detectors. The newer version, KAPTURE-2, was designed for more accurate sampling for pulse repetition rates up to \SI{2}{\giga \hertz}. The acquired data is processed by a FPGA and GPU architecture \cite{caselleKAP}.
The general structure of the board is shown in \autoref{fig:kapture}.

\begin{figure}[H]
	\centering
	\includegraphics[width = \textwidth]{chap/02-theory/img/kapture.tikz}
	\caption{General schema of KAPTURE (cf. \cite[p.2]{caselleKAP})}
	\label{fig:kapture}
\end{figure}

The pulse from the \SI{}{\tera \hertz} detector is fed into a power splitter, which splits the signal into four identical pulses and distributes them to four channels, consisting of a respective Track-And-Hold (TAH) unit and a 12-bit ADC@\SI{500}{\mega\sample\per\second}. The sampling time of each unit can be adjusted individually with a Picosecond Delay Chip with a resolution of \SI{3}{\pico \second} (maximal delay range: \SI{100}{\pico \second}). 
The clock signal is provided by KARA, which cleared from jitter by a Phase-Locked-Loop (PLL). The clean clock signal is distributed to the delay chips with a fan-out. \cite{caselleKAP}

This results in the sampling of the signal as shown in \autoref{fig:detector_signal}.
\begin{figure}[H]
	\centering
	\includegraphics[height = 0.3\textwidth, width = 0.6\textwidth]{chap/02-theory/img/detector_signal.tikz}
	\caption{Signal with sample points}
	\label{fig:detector_signal}
\end{figure}


\newpage 
\section{Optical Time Stretching Technique}
\subsection{Applications}

\newpage
\section{Analog-To-Digital-Converter}

\newpage

\section{New Readout System}
\subsection{Xilinx Zynq UltraScale+ RFSoC}
\subsection{Requirements}

\paragraph{Delay chip}
The necessary step size for the delay chips, when using 16 ADC@\SI{2}{\giga \sample \per \second} in time-interleaving mode, is: $\frac{\SI{2}{\giga \sample \per \second}}{16} = \SI{31}{\pico \second}$

%\begin{figure}[H]
%	\centering
%	\includegraphics[height = 0.3\textwidth, width = 0.6\textwidth]{chap/03-work/img/pll_karaMode.tikz}
%	\caption{PLL in "KARA-mode"}
%	\label{fig:pll}
%\end{figure}



