\section{Coherent Synchrotron Radiation}


\begin{figure}[H]
	\centering
	\includegraphics[width = 0.7\textwidth]{chap/02-theory/img/csr2.png}
	\caption{CSR \cite{rota2018}}
	\label{fig:csr}
\end{figure}



\begin{figure}[H]
	\centering
	\includegraphics[width = 0.8\textwidth]{chap/02-theory/img/spectrum.tikz}
	\caption{Electro-Magnetic spectrum}
	\label{fig:spectrum}
\end{figure}





\paragraph{KARA}
\begin{itemize}
	\item Located at the Karlsruhe Institute of Technology (KIT)
	\item Up to 184 electron packages (bunches) can be filled with a distance between two adjacent bunches of \SI{2}{\nano \second}
	\item Operated by the Institute of Beam Physics and Technology (IBPT)
\end{itemize}

\begin{figure}[H]
	\centering
	\includegraphics[width = 0.6\textwidth]{chap/02-theory/img/kara.png}
	\caption{Facility \cite{rota2018}}
	\label{fig:kara}
\end{figure}

\subsection{Micro-Bunching Instability}
\begin{figure}[H]
	\centering
	\includegraphics[width = 0.5\textwidth]{chap/02-theory/img/csr.png}
	\caption{Micro-bunching \cite{Bielawski2019}}
	\label{fig:microBunch}
\end{figure}

\newpage 
\section{Photonic time-stretch analog-to-digital converter}

"' In recent and future synchrotron radiation facilities, relativistic electron bunches with increasingly high charge density are needed for producing brilliant light at various wavelengths, from X-rays to terahertz. In such conditions, interaction of electron bunches with their own emitted electromagnetic fields leads to instabilities and spontaneous formation of complex spatial structures. Understanding these instabilities is therefore key in most electron accelerators. However, investigations suffer from the lack of non-destructive recording tools for electron bunch shapes. In storage rings, most studies thus focus on the resulting emitted radiation. Here, we present measurements of the electric field in the immediate vicinity of the electron bunch in a storage ring, over many turns. For recording the ultrafast electric field, we designed a photonic time-stretch analog-to-digital converter with terasamples/second acquisition rate. We could thus observe the predicted link between spontaneous pattern formation and giant bursts of coherent synchrotron radiation in a storage ring. "'  \cite{Bielawski2019}

\begin{figure}[H]
	\centering
	\includegraphics[width = 0.8\textwidth]{chap/02-theory/img/EO.png}
	\caption{Electro-Optical Time-Stretch Technique \cite{Bielawski2019}}
	\label{fig:eo}
\end{figure}


\newpage
\section{Analog-To-Digital-Converter}
\subsection{Sampling Theory}
\begin{itemize}
\item Nyquist-Criteria
\item Need for S/H
\end{itemize}
\subsection{AC errors}
\begin{itemize}
\item Quantization Noise
\item Equivalent Input Referred Noise
\item Noise-Free Code Resolution
\end{itemize}
\paragraph{Integral and Differential Nonlinearity Distortion} 

\paragraph{Dynamic Performance}

\begin{itemize}
\item Harmonic Distortion, Worst Harmonic, Total Harmonic Distortion (THD), Total Harmonic Distortion + Noise (THD + N)
\item Signal-to-Noise-and-Distortion Ratio (SINAD, or S/N + D), Signal-to-Noise ratio (SNR), Effective Number of Bits (ENOB)
\item Analog Bandwidth (Full-Power, Small-Signal)
\item Spurious Free Dynamic Range (SFDR)
\item Two-Tone Intermodulation Distortion, Multi-Tone Intermodulation Distortion
\item Noise Power Ratio (NPR)
\item Adjacent Leakage Ratio (ACLR)
\item Noise Figure
\item Setting Time, Overvoltage Recovery Time
\end{itemize}

\cite{walt}
\subsection{Interleaving}

\begin{itemize}
\item Net sample rate
\item Interleaving Spurs
\end{itemize}

\cite{mangrob}
%\begin{figure}[H]
%	\centering
%	\includegraphics[height = 0.3\textwidth, width = 0.6\textwidth]{chap/03-work/img/pll_karaMode.tikz}
%	\caption{PLL in "KARA-mode"}
%	\label{fig:pll}
%\end{figure}



\newpage
\section{RF/Microwave Design Basics}