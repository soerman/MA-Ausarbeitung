% todo: how to swap the abbreviation with long term

Analysis of events occurring in the time range of femtoseconds is necessary in many scientific experiments.
Continuous measurement of such events over a long period of time is desired in order to study the long-term evolution of such events.
With currently commercially available \gls{daq} this is not possible due to the high sampling rates required and the limited memory space of such systems. 

In this thesis, a first demonstrator of a new type of \gls{daq} has been developed to overcome these limitations.
The system consists of a high bandwidth front-end sampling card, mounted on a back-end card integrating a new generation of \gls{rfsoc} for continuous readout of the acquired samples. 
It is based on the photonic time-stretch method, using chirped laser pulses and dispersion in fibers to stretch the signal under study in time.
In this way, the signal can then be measured with a lower sampling rate than would usually be needed to measure events with a duration of several femtoseconds.
The name given to the system is \gls{theresa}.

The \gls{theresa} front-end sampling card integrates 16 sampling channels, each containing a \gls{tha} and an \gls{adc} (integrated in the \gls{rfsoc}) with individually programmable delay (step size \SI{11}{\pico \second}) in sampling time. 
With this setup, the so called time-interleaving technique can be implemented. 
This allows to sample the signal at higher sampling rate than one single data converter can achieve.

The design of the board allows it to be used either with the time-stretch setup or independently from it.
Furthermore, two different sampling modes are possible.
In single-channel mode one detector is connected to one sampling channel, therefore allowing sampling of up to 16 detectors at the same time with one sampling point per channel.
In the multi-channel mode, several channels are connected to one detector via power-splitter, therefore allowing multiple sampling points for one detector per channel by setting the delay times accordingly. 

High-speed \glspl{adc}, integrated in the \gls{rfsoc}, with 14-bit resolution and a sample rate of up to \SI{2.5}{\giga \sample\per\second} allow continuous sampling of the signal with high time resolution. 
Using the time-interleaving technique for all available \glspl{adc} results in an overall maximal achievable sample rate of \SI{40}{\giga\sample\per\second}. 
When used in combination with the time-stretch method and considering typical stretch-factors the signal can be sampled with a time resolution corresponding to hundreds of femtoseconds in the original signal.

The sampling card has furthermore been designed to fully exploit all the features of the \gls{rfsoc}, which integrates a processing unit together with a \gls{fpga}. 
The on-chip \gls{fpga} provides the possibility to flexibly adjust the firmware to user needs. 
Slow-control implemented in the \gls{fpga} takes care of programming the components on the sampling card, such as the delay chips.
The acquired samples are intermediately stored in the on-board \gls{ddr}, where they can be accessed by the high-speed data interface.
This interfaces allows transfer speeds over \SI{100}{\giga\bits\per\second}, are a crucial component for the high throughput of the large amount of data continuously generated by the data converters.
The processing unit can host for example an operating system and communicated with the firmware on the \gls{fpga}.
Over common periphery, such as Ethernet, the user can access and control the overall system.

With the evaluation tool provided by the manufacturer a quick set-up and measurement of key data converter performance characteristics (e.g. \gls{sinad} or \gls{sfdr}) is possible. 
Together with provided add-on cards, a first evaluation of the data converter performance has been possible.  %todo konkrete dinge sagen? 

The design of the sampling card was approved by the \gls{ipe} and the card is currently in production.
It has been shown, that the evaluation tool provided for the readout card allows to perform the necessary measurements (see \autoref{ssec:adc_charac}) for a quick characterization of the sampling card.
\gls{theresa} will then be commissioned and taken into operation, improving the research in various scientific fields, especially beam diagnostics at e.g. at \gls{kara}. 
There it can be used for studying \gls{csr}, in the far-field and near-field \gls{eo} setup, for study of fast laser dynamics and many other applications.
The selected \gls{fpga} is powerful and suitable for deploying \gls{ai} applications (i.e. reinforcement learning).
Therefore the system can also be used for interfacing with the Bunch-By-Bunch feedback at \gls{kara}.
In the context of the \gls{ultrasync} project, funded by \gls{anr} and \gls{dfg}, \gls{theresa} can be used to study the control of electron bunches in accelerators at \gls{kara} and \gls{soleil}. 

Therefore the newly developed \gls{theresa} system is an important step towards new usable \gls{thz} sources and can be deployed to improve beam diagnostics at several renowned facilities. 
  

%There is a disturbing lack of benches in \sout{Ramset Park} Campus North. I want to sit more!
