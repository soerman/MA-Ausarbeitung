\chapter*{Abstract}
In many physical experiments and applications, e.g. beam diagnostics, measuring events occurring in the time range of femtoseconds over a long period of time is necessary. 
This imposes a great technological challenge on \glspl{daq} and \glspl{adc} concerning sampling rate and memory space.
In order to relax the requirements on the acquisition systems, the photonic time-stretch technique is used to stretch the analog input signal in time.
In this way \glspl{daq} can be operated at lower sample rate than would be required without it (several \gls{thz}). 
Measuring the signal with commercial \glspl{daq}, e.g. storage oscilloscopes, still poses another challenge.
Due to the limited memory of such systems, continuous measurements at high sampling rate over a long period of time is not possible.
Therefore new concepts of \glspl{daq} based on the photonic time-stretch method need to be considered.

In this thesis, a first demonstrator of such a new photonic time-stretch based \gls{daq} system, \gls{theresa}, has been developed. 
The system consists of a high bandwidth front-end sampling card, mounted on a back-end readout card integrating a new generation of \gls{rfsoc} for readout and processing of the acquired samples.
%todo photonic time stretch erklären

First, the signal under study is stretched using chirped optical pulses and the chromatic dispersion in optical fibers.
The signal is measured with a photodetector and sampled by the front-end sampling card.
The front-end sampling card integrates 16 channels, each containing a \gls{tha} and \gls{adc} (integrated in the \gls{rfsoc}), the sampling time of which can be delayed individually. 
In this way the so-called time-interleaving method can be implemented, allowing for overall higher sampling rate.
The design of the board allows it to be used with the time-stretch method as well as independently from it.
Furthermore, the setup allows for different sampling modes.
In single-channel mode one detector is connected to one channel, allowing to acquire data from up to 16 detectors at the same time with one sampling point per channel.
In the multi-channel mode, several channels are connected to one detector via power-splitter, therefore allowing multiple samples per detector. %todo sample time

The \gls{rfsoc} on the back-end readout card integrates a processing unit and a \gls{fpga}. 
A firmware running on the \gls{fpga} is responsible for programming and controlling the components on the sampling card, as well as continuously acquiring the samples and sending it to the following processing system via high-speed connections.
The processing unit, hosting e.g. an operating system, allows for the user to control and monitor the overall system via common periphery.

The 16 high-speed \glspl{adc} integrated in the \gls{rfsoc} are capable of a sampling rate of up to \SI{2.5}{\giga \sample \per \second}.
Using the time-interleaving technique for all \glspl{adc} results in an maximal achievable sample rate of \SI{40}{\giga \sample \per \second}.  %todo sample rate of overall system
Using the system with the time-stretch setup a time resolution in the range of hundred of femtoseconds is possible, considering currently achievable time-stretch factors.
\gls{theresa} is therefore suitable to be used in beam diagnostics, e.g. at the \gls{kara}.
