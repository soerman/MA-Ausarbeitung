\chapter*{Abstract}

Analysis of events occurring in the range of femtoseconds is desired in many scientific experiments.
The high temporal resolution needed for measuring such events imposes a great technological challenge for \glspl{daq} and \glspl{adc}.
In order to relax the requirements on the acquisition systems, the so-called optical time-stretch technique is used to stretch the analog input signal in time.
In this way, data converters at relatively moderate sample rate can be used.
Measuring the signal with commercial \glspl{daq}, such as real-time oscilloscope, still poses another challenge.
Due to the limited acquisition time windows of such systems, continuous measurements at high sampling rate and time resolution over a long period of time is not possible.
In applications, where measurements of long-term evolution of the ultra-fast events with high temporal resolution is necessary, this is a large limitation.
Therefore new concepts of \gls{daq} based on the time-stretch method need to be considered.

In this thesis, a first demonstrator of such a new photonic time-stretch based \gls{daq} system was developed.
The system consists of a high bandwidth front-end sampling card, mounted on a back-end readout card integrating a new generation of \gls{rfsoc} for readout of the acquired samples. 

The front-end sampling card integrates 16 sampling channels, each containing a \gls{tha}. 
The sampling time of these \glspl{tha} can be delayed individually in steps of \SI{11}{\pico \second}, covering a range up to \SI{11.2}{\nano \second}.
In this way the so-called time-interleaving method can be implemented to sample the signal at a higher rate thant that normally possible due to the Nyquist theorem.
The design of the board allows it to be used with the time-stretch method as well as independently from it.
Furthermore, the setup allows for different sampling modes.
In single-channel mode one detector is connected to one sampling channel, therefore allowing to acquire data from up to 16 detectors at the same time with one sampling point per channel.
In the ``multi-channel'' mode, several channels (up to 16) are connected to one detector via power splitter, therefore allowing multiple sampling points for one detector.

High-speed \glspl{adc}, integrated in the \gls{rfsoc}, with 14-bit resolution and a sample rate of up to \SI{2.5}{\giga \sample \per \second} allow continuous sampling of the signal with high temporal resolution. 
Using the time-interleaving technique for all sixteen \glspl{adc} results in an overall maximal achievable sample rate of \SI{40}{\giga \sample \per \second}.  
When used in combination with the time-stretch technique and considering currently achievable time-stretch factors, a time resolution in the range of hundred of femtoseconds is possible.

The \gls{rfsoc} on the back-end readout card integrates a processing unit and a \gls{fpga}. 
A firmware running on the \gls{fpga} is responsible for programming and controlling the components on the sampling card, as well as collecting the acquired samples and sending it to the following processing system via high-speed connections.
The processing unit, hosting e.g. an operating system or a standalone application, allows for the user to control and monitor the overall system via common periphery, e.g. Ethernet.

The name given to the system is THERESA, an acronym for ``Terahertz Readout Sampling''.
\chapter*{Zusammenfassung}
\chapter*{Résumé}