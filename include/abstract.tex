\chapter*{Abstract}
In many physical experiments and applications, e.g. beam diagnostics, measuring events occurring in the time range of femtoseconds over a long period of time is necessary. 
This imposes a great technological challenge on \glspl{daq} and the \glspl{adc} in them concerning sampling rate and memory space.
In order to relax the requirements on the acquisition systems, the photonic time-stretch technique can be used to stretch the analog input signal in time.
This way, the \glspl{adc} can be operated at a lower sample rate than the several \si{\THz} that would be required without it. 
Measuring the signal with commercial \glspl{daq}, e.g. digital storage oscilloscopes, there still is another challenge.
Due to the limited memory of such systems, continuous measurements at high sampling rates over a long period of time is not possible.
Therefore new concepts for photonic time-stretch based \glspl{daq} are needed.

In this thesis, a first demonstrator of such a novel photonic time-stretch based \gls{daq} system, called \gls{theresa}, has been developed.
The development phase includes researching necessary components, designing the front-end sampling cards circuit and \gls{pcb} and writing a firmware for testing purposes.
The designed system consists of a high bandwidth front-end sampling card, mounted on a back-end readout card integrating a new generation of \gls{rfsoc} for readout and processing of the acquired samples.

The input signal (e.g. \si{\tera \hertz} pulse)is first stretched using chirped optical pulses and exploiting the chromatic dispersion of two optical fibers.
The stretched signal is then detected by a photodetector.
The analog signal is then sampled with the 16 channel front-end sampling card.
Each channel contains a \gls{tha} and a \gls{adc} (integrated in the \gls{rfsoc}) capable of sampling at \SI{2.5}{\giga \sample \per \second}.
The sampling time of each \gls{tha} can be delayed in time individually. 
Thereby the so-called time-interleaving method can be implemented, allowing for overall higher sampling rate.
The design of the board allows it to be used with the time-stretch method as well as independently from it.
Furthermore, the setup allows for different sampling modes.
In single-channel mode one detector is connected to one channel, allowing to acquire data from up to 16 detectors at the same time with one sampling point per channel.
In the multi-channel mode, several channels are connected to one detector via a power-splitter, therefore allowing multiple samples per detector.

The \gls{rfsoc} on the back-end readout card integrates a processing unit and a \gls{fpga}. 
A custom firmware running on the \gls{fpga} is responsible for configuring and controlling the components on the sampling card, as well as continuously acquiring the samples and relaying them to the following processing system via high-speed on-chip connections.
If the processing unit hosts an operating system, it allows the user to control and monitor the overall system via common periphery or over a network connection.

Using the time-interleaving technique for all \glspl{adc} results in an overall achievable sample rate of \SI{40}{\giga \sample \per \second}.  %todo sample rate of overall system
Using the system with the time-stretch setup, a time resolution in the range of hundreds of femtoseconds is possible, considering the currently achievable time-stretch factors.
\gls{theresa} therefore achieves the requirements set by particle accelerator facilities and can be a viable tool to be used in beam diagnostics, e.g. at \gls{kara}.
